\chapter{\IfLanguageName{dutch}{Stand van zaken}{State of the art}}%
\label{ch:stand-van-zaken}

% Tip: Begin elk hoofdstuk met een paragraaf inleiding die beschrijft hoe
% dit hoofdstuk past binnen het geheel van de bachelorproef. Geef in het
% bijzonder aan wat de link is met het vorige en volgende hoofdstuk.

% Pas na deze inleidende paragraaf komt de eerste sectiehoofding.

%Dit hoofdstuk bevat je literatuurstudie. De inhoud gaat verder op de inleiding, maar zal het onderwerp van de bachelorproef *diepgaand* uitspitten. De bedoeling is dat de lezer na lezing van dit hoofdstuk helemaal op de hoogte is van de huidige stand van zaken (state-of-the-art) in het onderzoeksdomein. Iemand die niet vertrouwd is met het onderwerp, weet nu voldoende om de rest van het verhaal te kunnen volgen, zonder dat die er nog andere informatie moet over opzoeken \autocite{Pollefliet2011}.

%Je verwijst bij elke bewering die je doet, vakterm die je introduceert, enz.\ naar je bronnen. In \LaTeX{} kan dat met het commando \texttt{$\backslash${textcite\{\}}} of \texttt{$\backslash${autocite\{\}}}. Als argument van het commando geef je de ``sleutel'' van een ``record'' in een bibliografische databank in het Bib\LaTeX{}-formaat (een tekstbestand). Als je expliciet naar de auteur verwijst in de zin (narratieve referentie), gebruik je \texttt{$\backslash${}textcite\{\}}. Soms is de auteursnaam niet expliciet een onderdeel van de zin, dan gebruik je \texttt{$\backslash${}autocite\{\}} (referentie tussen haakjes). Dit gebruik je bv.~bij een citaat, of om in het bijschrift van een overgenomen afbeelding, broncode, tabel, enz. te verwijzen naar de bron. In de volgende paragraaf een voorbeeld van elk.

%\textcite{Knuth1998} schreef een van de standaardwerken over sorteer- en zoekalgoritmen. Experten zijn het erover eens dat cloud computing een interessante opportuniteit vormen, zowel voor gebruikers als voor dienstverleners op vlak van informatietechnologie~\autocite{Creeger2009}.

%Let er ook op: het \texttt{cite}-commando voor de punt, dus binnen de zin. Je verwijst meteen naar een bron in de eerste zin die erop gebaseerd is, dus niet pas op het einde van een paragraaf.

%\lipsum[7-20]

\acrfull{ip} is het fundament van elk gestructureerd, goed functionerend en veilig netwerk. Het geeft de mogelijkheid efficiënt gegevens te routeren, netwerken te verdelen in meer beheersbare eenheden, toegang te beperken tot gevoelige data of systemen, services te identificeren en het oplossen van netwerkproblemen \autocite{Postel1981}. Dit hoofdstuk legt uit wat \acrfull{dns} en \acrfull{dhcp} is, waarom \acrshort{ipam} helpt bij het beheren van \acrshort{ip} netwerken en waarom \acrshort{http} nodig is om te communiceren met EIP. 

\subsection{DNS}
\textcite{Mockapetris1987} schrijft dat \acrshort{dns} een systeem is dat \textit{resource records} gebruikt om onder andere vertalingen te voorzien tussen domeinnamen en \acrshort{ip}-adressen. Als voorbeeld kan je via de browser naar google surfen via het \acrshort{ip}-adres \textit{142.251.36.35} of via domeinnaam \textit{www.google.be}.

Zoals beschreven door \textcite{Mockapetris1987} voorziet \acrshort{dns} meerdere types resource records die netwerkbeheerders kunnen meegeven: 
\begin{itemize}
    % TODO beschrijven wat een resource record is
    \item \textbf{A}: Dit resource record beschrijft een host adres. 
    Vb. \textit{”server1.voorbeeld.com. IN A 192.168.1.1”} maakt de vertaling zodat het toestel met de domeinnaam \textit{server1.voorbeeld.com} bereikbaar is zowel via het \acrshort{ip}-adres \textit{192.168.1.1} als via de domeinnaam. 
    \item \textbf{CNAME}: Dit resource record beschrijft de canonieke naam van een host, het wordt gebruikt om een alias of subdomein naar het hoofddomein door te verwijzen. Vb. \textit{"www.voorbeeld.com. IN CNAME server1.voorbeeld.com"} zorgt dat server1 ook bereikbaar is via "\textit{www.voorbeeld.com}".
    \item \textbf{MX}: Dit resource record is een \textit{mail exchange} record en wordt gebruikt om aan te geven welke mailservers verantwoordelijk zijn voor het ontvangen van mails binnen een domein. vb. \textit{"voorbeeld.com. IN MX 10 mailserver.voorbeeld.com"} geeft de \acrshort{dns} server mee welke server de mailserver is.
    \item \textbf{NS}: Dit resource record is een \textit{name server} record, het beschrijft welke \acrshort{dns}-servers verantwoordelijk zijn voor het beheren van \acrshort{dns}-informatie voor een domein. Vb. \textit{"voorbeeld.com. IN NS dns1.voorbeeld.com"} verwijst naar \textit{dns1} als \acrshort{dns}-server voor het domein “\textit{voorbeeld.com}”.
    \item \textbf{PTR}: Dit resource record is een \textit{Pointer} record, het wordt gebruikt om via \acrshort{ip} een vertaling te vragen aan de \acrshort{dns}-server in plaats van via de naam.
    \item \textbf{SOA}: Dit resource record is een \textit{Start of Authority} record die belangrijke informatie bevat over de zone, zoals welke de primaire \acrshort{dns}-server, contactpersonen, etc. zijn.
\end{itemize}