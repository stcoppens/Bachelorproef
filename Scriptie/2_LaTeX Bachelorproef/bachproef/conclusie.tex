%%=============================================================================
%% Conclusie
%%=============================================================================

\chapter{Conclusie}%
\label{ch:conclusie}

Deze bachelorproef richt zich op het adresseren van de inefficiënties en mogelijke fouten bij het handmatig beheren van IP-registraties binnen het netwerk van UGent. Dankzij de ontwikkeling van een nieuwe registratietool, bestaande uit de combinatie van Python-scripts, een gebruiksvriendelijke webinterface, en DDI-systeem EfficientIP, worden aanzienlijke verbeteringen gerealiseerd in termen van efficiëntie, tijdsbesparingen en het verminderen van fouten.
Er wordt een antwoord gegeven op de uitdagingen uit sectie \ref{uitdagingen}:
\begin{itemize}[]
    \item \textbf{Tijd}: Aangezien er een aparte pagina op de registratietool is gemaakt waar netwerkbeheerders de gemaakte registraties of wijzigingen kunnen bekijken en goed- of afkeuren, wat vervolgens automatisch wordt verwerkt, zullen ze tijd besparen. Dankzij de metingen die gedaan zijn in hoofdstuk \ref{ch:netadmin-voorbereidingen} is nu duidelijk dat de netwerkbeheerders hier gemiddeld 17 minuten en 54 seconden per dag zullen besparen.
    \item \textbf{Schaalbaarheid}: EfficientIP stelt UGent in staat om IP-adressen efficiënter toe te wijzen en te beheren, waardoor conflicten worden voorkomen en de uitrol van nieuwe services wordt versneld. Daarnaast automatiseert het routinetaken en ondersteunt het de implementatie van IPv6, waardoor UGent gemakkelijk kan meeschalen met groeiende netwerken en voldoet aan moderne standaarden.
    \item \textbf{Consistentie}: De registratietool maakt het mogelijk voor gebruikers om eenvoudig IP-registraties aan te vragen door middel van een intuïtieve interface, waarbij de complexiteit van het selecteren van de juiste subnetten wordt verminderd door gebruik te maken van keuzelijsten op basis van campus- en gebouwen, correcte subnetnamen en achterliggende FI-codes. Dit leidt tot een vereenvoudigde aanpak en heeft het potentieel om menselijke fouten te verminderen.  
    \item \textbf{Beveiliging}: De registratietool houdt bij welke netwerkbeheerder wanneer een registratie heeft goedgekeurd. Deze informatie wordt samen met de goedgekeurde registratie naar EfficientIP verzonden, waardoor altijd bekend is wie de laatste wijziging heeft gedaan. Bovendien wordt alle netwerkdata opgeslagen in een geëncrypteerde database, wat de weerbaarheid tegen externe bedreigingen verhoogt.
\end{itemize}

\section{Verder onderzoek}
Deze registratietool geeft meerdere mogelijkheden voor verdere verbeteringen, zoals het implementeren van een API voor systeembeheerders, het mogelijk maken van bulkregistraties voor gebruikers, en implementatie van IPv6, die toekomstig onderzoek rechtvaardigen.

In essentie levert dit onderzoek een waardevolle bijdrage aan het domein van netwerkbeheer door het ontwikkelen van een innovatieve aanpak voor het beheren en toewijzen van netwerkadressen. Deze aanpak biedt niet alleen tastbare voordelen voor de netwerkbeheerders van UGent, maar heeft ook het potentieel om als model te dienen voor vergelijkbare organisaties die worstelen met vergelijkbare uitdagingen in netwerkbeheer.


