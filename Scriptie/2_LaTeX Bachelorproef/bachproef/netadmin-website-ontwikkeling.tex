%%=============================================================================
%% netadmin-website-ontwikkeling
%%=============================================================================

\chapter{Netadmin - Website ontwikkeling}%
\label{ch:netadmin-website-ontwikkeling}

\section{Databank}
\label{databank}

\section{Pagina: Zoek registraties}
\label{zoek-registraties}
\subsection{Front-end}
\subsection{Back-end}
%- getHosts: 1 of meerdere hostregistraties ophalen van EIP en deze beschikbaar stellen voor frontend, al dan niet met vakgroep filter


\section{Pagina: Nieuwe registratie}
\label{nieuwe-registratie}
\subsection{Front-end}
\subsection{Back-end}
%- getSubnets: De gebruiker van de frontend kan een campus en gebouw uit de lijst van alle UGent campussen en gebouwen kiezen, op basis van deze 2 parameters wordt de unieke FI-code van het gebouw opgehaald. Elk subnet is gelinkt aan 1 of meerdere FI-codes, op basis van deze FI-code halen we de nodige informatie op van alle subnetten die de gevraagde FI-code bevatten.
%- getAvailableIP: Dit script gebruikt het unieke subnet-ID van EfficientIP (deze kennen we dankzij getSubnets) om alle vrije, beschikbare IP-adressen op te halen van het gevraagde subnet.

\section{Wijzigingen verwerken}
\label{wijzigingen-verwerken}
%- processChanges: Dit script haalt in de MySQL-database die voor dit project is opgezet alle hosts op die voldoen aan volgende criteria: netadmin_handled = 0 (neen), EN netadmin_approved = 1 (ja). Dit betekent dat we enkel de hosts uitlezen die nog niet verwerkt zijn maar die wel reeds goedgekeurd zijn door de netadmin. De nodige API-URL's worden gemaakt om de aanpassingen te maken in EIP, indien de failsave niet is overschreden, worden aangemaakte URL's gebruikt om de aanpassingen live te maken. Als elke API-call gelukt is (dit weten we dankzij het antwoord die EIP stuurt bij elke call), update het script het netadmin_handled veld van die host in de MySQL database.

% TODO: subsectie per pagina
% TODO: subsubsectie bespreking frontend en backend van elke subsectie
% TODO: bij bespreking backend duidelijke uitleg geven over de werking van elk script
% TODO: bespreking databank en interactie
\lipsum[76-80]


