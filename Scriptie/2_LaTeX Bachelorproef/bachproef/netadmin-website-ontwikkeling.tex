%%=============================================================================
%% netadmin-website-ontwikkeling
%%=============================================================================

\chapter{Netadmin - Website ontwikkeling}%
\label{ch:netadmin-website-ontwikkeling}
Dit hoofdstuk zal enkele belangrijke componenten voor de website beschrijven. Beginnende bij de databank die data persistent zal bewaren, gevolgd door de pagina's op de website zelf met achterliggende scripts. Om dan te eindigen bij de scripts die de data uit de databank zal ophalen en zal doorsturen naar EfficientIP.

\section{Databank}
\label{databank}
Tijdens de ontwikkelingsfase van de website wordt gebruikt gemaakt van een MySQL-databank in de testomgeving van UGent. Er wordt beslist om binnen deze databank drie tabellen aan te maken: \textit{new\_hosts}, \textit{change\_hosts} en \textit{remove\_hosts}. Elke tabel zal door de frontend gevuld worden op basis van de door de gebruiker ingevulde velden en data die meegegeven wordt door scripts.
De tabel \textit{change\_hosts} zal steeds twee versies van elke registratie bevatten, de originele versie van de host, en de gewijzigde versie van de host.

\section{Toegangsregels}
\label{toegangsregels}
Voor het ontwikkelen van de website wordt voor de toegangsregels een onderscheid gemaakt tussen enkele subnetten: 
\begin{itemize}
    \item Klassieke Data (data): Binnen deze subnetten komen de standaard registraties.
    \item Co-locatie (colo): Hierin komen registraties van toestellen die op verschillende locaties gebruikt worden, deze subnetten beperken zich dus niet tot één gebouw.
    \item Important: In deze subnetten komen belangrijke registraties voor toestellen zoals camera's, alarmen, en dergelijke.
    \item Overige: Dit zijn subnetten zoals voice (voor telefonie), wireless (wifi accesspoints), en andere.
\end{itemize} 
De belangrijkste regel is dat enkel personeel toegang met een UGent-account mag hebben tot deze informatie.
Echter mag niet elk personeelslid toegang hebben tot alle subnetten, daarom worden er vijf verschillende types gebruikersgroepen gemaakt met elk hun eigen toegangsregels:
\begin{itemize}
    \item Gewone gebruikers: Deze personeelsleden mogen registraties inzien en maken voor data- en colo-subnetten van hun eigen vakgroep.
    \item Multi-vakgroep gebruikers: Dit zijn bijvoorbeeld facultaire systeembeheerders die ook voor andere vakgroepen registraties moeten kunnen maken. Ze kunnen dus registraties inzien en maken voor data- en colo-subnetten voor hun eigen vakgroep en een beperkt aantal andere vakgroepen. Deze lijst hangt af per gebruiker en wordt manueel beheerd.
    \item CA70: Dit zijn gebruikers die tot vakgroepen CA70 en CA80 behoren. Ze mogen registraties, binnen hun eigen vakgroep, maken voor data-, colo- en important subnetten.
    \item CA60: Gebruikers die tot deze vakgroep behoren mogen registraties maken en inzien voor alle subnetten van alle vakgroepen. Daarnaast mogen ze eveneens registraties maken in naam van andere personeelsleden en kunnen ze ook informatie zien bij de registraties, dit is informatie zoals gebruikte ACL-lijnen en aliasen.
    \item netadmin: Deze groep van gebruikers heeft dezelfde rechten als CA60. Daarnaast hebben ze ook de exclusieve toegang tot de pagina waarop registraties kunnen bewerkt en goed- of afgekeurd worden.
\end{itemize}

\section{Pagina: Zoek registraties}
\label{zoek-registraties}
Deze pagina laat ingelogde gebruikers toe om, op basis van de toegangsregels, registraties te zoeken. De resultaten kan de gebruiker dan wijzigen of verwijderen.
\subsection{Front-end}
De PHP-webpagina bevat enkel een zoekveld waarin de ingelogde gebruiker registraties kan zoek op basis van de toegestane toegangsregels en volgende zoektermen:
\begin{itemize}
    \item IP-adres: Indien het een geldig IP-adres is zal de gebruiker de eventuele registratie die tot dit IP behoort terugvinden.
    \item MAC-adres: Zoals bij IP-adres zal dit de registratie(s) weergeven die dit MAC-adres bevatten.
    \item IP-ID: Dit maakt het mogelijk om te zoeken naar registraties op basis van het ID waaronder de registratie gekend is binnen EfficientIP.
    \item Vakgroep: Dit geeft alle registraties weer die tot deze vakgroep behoren.
    \item FI-code: Dit zal alle registraties weergeven die onder deze unieke gebouwcode behoren.
    \item Campus: Hiermee kan men alle registraties die tot een bepaalde campus behoren opvragen.
    \item Gebouw: alle registraties die tot dit gebouw behoren.
    \item Contactpersoon: Dit kan op basis van “voornaam naam” of op basis van e-mail. Hiermee kan je alle registraties opvragen die deze persoon als contactpersoon bevatten.
    \item FQDN: FQDN staat voor Fully Qualified Domain Name, indien een gebruiker dit gebruikt zal die de registratie met deze FQDN als resultaat te zien krijgen.
    \item hostnaam: Dit heeft hetzelfde resultaat als bij het zoeken op basis van FQDN.
\end{itemize}
Of een gebruiker nu 1 of meerdere registraties opzoekt, elk resultaat wordt onder elkaar samengevat weergegeven. Indien men op een resultaat klikt kan men deze openklappen en zal men alle informatie van deze registratie kunnen inzien. Naast de informatie van de registratie zullen er ook twee knoppen beschikbaar zijn, 1 knop om de registratie te bewerken, en 1 om de registratie te verwijderen.
Elke knop zal een pop-up openen voor meer informatie en/of de gevraagde actie te bevestigen.

\subsection{Back-end}
Voor de back-end van deze pagina is het python-script \textit{getHosts.py} geschreven. Nadat alle nodige logging-acties genomen zijn, zal het script de meegegeven argumenten controleren  op aantallen en weggeschrijven naar een variabele. Als dit allemaal correct verloopt worden de inloggegevens voor EfficientIP opgehaald. Hierna wordt gekeken wat voor zoekterm is meegegeven als argument, op basis van deze beslissing wordt dan contact gemaakt met EfficientIP via de juiste API-call. Naast de zoekterm kan het script ook 1 of meerdere vakgroepen als argument opvangen. Indien dit het geval is, zullen deze vakgroepen gebruikt worden om de resultaten te filteren die de gebruiker kan opvragen.

Indien EfficientIP een antwoord geeft dat er geen registraties zijn op basis van de zoekterm wordt dit weggeschreven in de log en eindigt het script. Indien EfficientIP wel registraties teruggeeft wordt een onderscheid op het aantal registraties.
Indien er meer dan 1 registratie wordt weergegeven, wordt enkel de FQDN, IP, MAC en IP-ID van elke registratie in een lijst weggeschreven naar de output van het script. Als er maar 1 registratie is, wordt de host volledige verwerkt. Deze verwerking zal nog extra API-calls naar EfficientIP sturen om bijvoorbeeld de aliasen die tot de registratie behoren op te halen.

Deze manier van werken werd gekozen omdat sommige zoektermen (zoals vakgroep of campus) een groot aantal registraties zal weergeven waardoor het allemaal lang kan duren voordat de gebruiker zijn resultaten krijgt. Indien een gebruiker dan een registatie openklikt zal het script nogmaals de host opzoeken op basis van IP-ID, waardoor die maar 1 resultaat zal moeten verwerken en weergeven.

\section{Pagina: Nieuwe registratie}
\label{nieuwe-registratie}
Deze pagina laat ingelogde gebruikers toe om nieuwe registraties te maken. De velden die worden weergegeven hangen af van de toegangsregels voor de gebruiker.

\subsection{Front-end}
Een formulier wordt weergegeven waarbij bij sommige velden een keuze gemaakt moet worden op basis van een oplijsting.
Een voorbeeld daarvan is locatie, waarbij de gebruiker een campus en gebouw moet kiezen. Eens deze gekozen zijn zal achterliggend een script in werking treden die op basis van de FI-code van het gebouw alle subnetten zal ophalen die in dat gebouw gebruikt worden.
Deze oplijsting van subnetnamen wordt gefilterd op basis van de toegangsregels om dan als lijst gepresenteerd te worden aan de gebruiker, zodat die het juiste subnet kan kiezen. 

\subsection{Back-end}
Voor het ophalen van de subnetten op basis van FI-code is het python-script \textit{getSubnets.py} geschreven. Ook dit script schrijft een log weg, controleert de meegegeven argumenten op aantallen en haalt de inloggegevens voor EfficientIP op. Dit script accepteert twee argumenten: campus en gebouw. Op basis van een JSON-bestand waarin alle campussen en gebouwen en de daarbij horende FI-codes staan, wordt de juist FI-code opgehaald. Daarna zal het script via een API-call alle subnetten die tot deze FI-code behoren, ophalen binnen EfficientIP. Voor elk subnet in de resultaten wordt de volgende informatie weggeschreven naar de output van het script:
\begin{itemize}
    \item Name: Naam van het subnet.
    \item Mask: Netwerkmasker van het subnet, hiervoor wordt een vertaling gedaan van de subnet grootte naar het netwerkmasker.
    \item Broadcast: Het broadcastadres van het subnet, dit is het laatste adres van het netwerk.
    \item Address: Dit is het eerste adres van het netwerk gevolgd door de prefix van het netwerk. Dit geeft een duidelijk beeld wat het bereik is van dit subnet.
    \item ID: Dit is het ID waaronder het subnet gekend is binnen EfficientIP.
    \item Dot1x: Hierin wordt de informatie weergegeven die EfficientIP voor dot1x bevat voor dit subnet.
    \item Gateway: Dit is de gateway van het subnet, het adres van de router die elke host binnen het netwerk moet gebruiken om buiten het netwerk te communiceren.
    \item Subdomain: Het eventuele subdomein binnen het UGent domein waartoe dit subnet behoort, indien er 1 is.
    \item Nameservers: Een oplijsting van de twee DNS-servers die binnen dit subnet gebruikt worden voor DNS-vertalingen.
    \item data: Dit veld bevat True of False, naargelang of data in de naam van het subnet zit.
    \item important: Dit veld bevat True of False, naargelang of important in de naam van het subnet zit.
    \item colo: Dit veld bevat True of False, naargelang of \_colo in de naam van het subnet zit.
\end{itemize}
Indien er geen subnetten zijn die tot de gevraagde FI-code behoren zal dit in de log terecht komen.

\section{Pagina: Aanvragen}
\label{aanvragen}
Deze pagina die exclusief voor de netwerkbeheerders is, wordt gebruikt om aanvragen bij te werken, en goed of af te keuren.

\subsection{Front-end}
Op deze pagina is een overzicht van alle registraties te zien, bij elke registratie in de MySQL-tabellen zijn twee velden die aangeven wat de status van een registratie is wat betreft: goedkeuring netwerkbeheerder, en wat de status is met EfficientIP. Deze pagina geeft een overzicht van alle registraties die nog niet goedgekeurd zijn door de netwerkbeheerders en die nog niet verwerkt zijn en dus nog niet op EfficientIP staan. Nieuwe registraties zullen nog geen IP-adres hebben, deze kunnen de netwerkbeheerders automatisch laten aanvullen door op de daarvoor gemaakt knop te klikken.
Netwerkbeheerders zullen hier registraties kunnen aanpassen, goedkeuren, afkeuren en, al dan niet, commentaar toevoegen.

\subsection{Back-end}
Netwerkbeheerders kunnen via een knop op deze pagina IP-adressen ophalen voor elke nieuwe registratie die nog geen IP-adres bevat.
Hiervoor wordt het python-script \textit{getAvailableIP.py} gebruikt, dit script accepteert 2 argumenten: het subnet-ID  en dot1x. Allebei info die wordt meegegeven bij het ophalen van elk subnet (zie back-end in sectie \ref{nieuwe-registratie}). Nadat de logging is opgezet, de config is uitgelezen en de argumenten gecontroleerd zijn, worden uit de config de inloggegevens opgehaald voor zowel EfficientIP als MySQL. Er wordt een API-call gemaakt naar EfficientIP op basis van het meegegeven subnetID. Indien er geen resultaten zijn, wordt dit in de log geplaatst en zullen de netwerkbeheerders het subnet moeten opkuisen.
Indien er wel vrije IP-adressen in het subnet beschikbaar zijn, wordt er nagekeken of het IP-adres in een pool zit, wat voor pool, of dot1x actief is en of het IP-adres al in gebruik is door een andere registratie en dus reeds is toegewezen in de MySQL tabel new\_hosts.

\section{Wijzigingen verwerken}
\label{wijzigingen-verwerken}
Er worden eveneens drie scripts geschreven die op de webserver geplaatst zullen worden en die periodiek zullen draaien.
Deze drie python-scripts zijn \textit{processNewHosts.py}, \textit{processChangeHosts.py} en \textit{processDeleteHosts.py}.

In tegenstelling tot de voorgaande scripts worden er geen argumenten meegegeven, deze scripts zetten logging op, lezen de config uit en beginnen elk met het verwerken van alle onverwerkte, goedgekeurde registraties in de tabel die daarvoor voorzien is. Bijvoorbeeld processNewHosts.py kijkt in de database enkel naar de tabel new\_hosts.
Alle URL'en die nodig zijn om de host aan te maken/wijzigen of verwijderen via een API-call worden per host in een lijst geplaatst.
Nadat alle URL'en gemaakt zijn, wordt geteld hoeveel dit er zijn, en wordt gekeken of deze boven een bepaald getal uitkomen.
Dit getal is een zogenaamde \textit{failsave}, indien er teveel URL'en zijn is er iets foutgelopen en wordt dit gelogged, en worden er dus geen API-calls gemaakt naar EfficientIP.
Als de failsave niet afgaat, worden alle URL'en per hostregistratie afgegaan en naar EfficientIP verzonden. 
Indien elke call voor een registatie succesvol verloopt (dit weet het script op basis van het antwoord van EfficientIP), zal het script contact maken met de MySQL database en de waarde aan te passen om aan te geven dat de registratie verwerkt is.



