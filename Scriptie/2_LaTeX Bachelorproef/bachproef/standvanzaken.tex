\chapter{\IfLanguageName{dutch}{Stand van zaken}{State of the art}}%
\label{ch:stand-van-zaken}

% Tip: Begin elk hoofdstuk met een paragraaf inleiding die beschrijft hoe
% dit hoofdstuk past binnen het geheel van de bachelorproef. Geef in het
% bijzonder aan wat de link is met het vorige en volgende hoofdstuk.

% Pas na deze inleidende paragraaf komt de eerste sectiehoofding.

%Dit hoofdstuk bevat je literatuurstudie. De inhoud gaat verder op de inleiding, maar zal het onderwerp van de bachelorproef *diepgaand* uitspitten. De bedoeling is dat de lezer na lezing van dit hoofdstuk helemaal op de hoogte is van de huidige stand van zaken (state-of-the-art) in het onderzoeksdomein. Iemand die niet vertrouwd is met het onderwerp, weet nu voldoende om de rest van het verhaal te kunnen volgen, zonder dat die er nog andere informatie moet over opzoeken \autocite{Pollefliet2011}.

%Je verwijst bij elke bewering die je doet, vakterm die je introduceert, enz.\ naar je bronnen. In \LaTeX{} kan dat met het commando \texttt{$\backslash${textcite\{\}}} of \texttt{$\backslash${autocite\{\}}}. Als argument van het commando geef je de ``sleutel'' van een ``record'' in een bibliografische databank in het Bib\LaTeX{}-formaat (een tekstbestand). Als je expliciet naar de auteur verwijst in de zin (narratieve referentie), gebruik je \texttt{$\backslash${}textcite\{\}}. Soms is de auteursnaam niet expliciet een onderdeel van de zin, dan gebruik je \texttt{$\backslash${}autocite\{\}} (referentie tussen haakjes). Dit gebruik je bv.~bij een citaat, of om in het bijschrift van een overgenomen afbeelding, broncode, tabel, enz. te verwijzen naar de bron. In de volgende paragraaf een voorbeeld van elk.

%\textcite{Knuth1998} schreef een van de standaardwerken over sorteer- en zoekalgoritmen. Experten zijn het erover eens dat cloud computing een interessante opportuniteit vormen, zowel voor gebruikers als voor dienstverleners op vlak van informatietechnologie~\autocite{Creeger2009}.

%Let er ook op: het \texttt{cite}-commando voor de punt, dus binnen de zin. Je verwijst meteen naar een bron in de eerste zin die erop gebaseerd is, dus niet pas op het einde van een paragraaf.

%\lipsum[7-20]
Dit hoofdstuk begint met het verklaren van basis netwerkbegrippen zoals IP en subnetten, gaat vervolgens over naar meer geavanceerde begrippen zoals DNS, DHCP, IPAM, DDI en HTTP/HTTPS, en eindigt met een vergelijkend onderzoek, waarbij overeenkomsten met een soortgelijk project worden besproken en lessen die daaruit kunnen worden getrokken.


\section{IT Netwerken: fundamenten}
In dit gedeelte wordt de essentiële basis van IP en subnetten uiteengezet. Eerst worden de fundamentele concepten van IP-adressering en subnetmaskers besproken, die cruciaal zijn voor het begrijpen van netwerken. Vervolgens wordt ingegaan op hoe subnetten worden toegepast om netwerken te organiseren en efficiënter te beheren.

\subsection{IP: Internet Protocol}
% TODO: Bron zoeken om te staven wat ik schrijf over IP
\acrfull{ip} is een set van regels die bepalen hoe een computer data zal versturen naar een andere computer. Het is het fundament van elk gestructureerd, goed functionerend en veilig netwerk. IP maakt efficiënte gegevensoverdracht mogelijk, verdeelt netwerken in beheersbare eenheden, beperkt toegang tot gevoelige informatie of systemen, identificeert services en helpt bij het oplossen van netwerkproblemen \autocite{Postel1981}. Een IP-adres is volgens \textcite{Postel1981} een belangrijk mechanisme van het IP. Het is een uniek identificatienummer dat aan elk apparaat is toegewezen, dat verbonden is met een computernetwerk dat het IP gebruikt voor communicatie. Dit adres helpt andere computers om te weten waar ze informatie naartoe moeten sturen wanneer ze communiceren via een netwerk.
Er zijn twee versies van IP-adressen, IPv4 en IPv6, waarvan de eerste versie bestaat uit vier groepen cijfers gescheiden door punten en de tweede versie uit acht groepen hexadecimale cijfers, gescheiden door dubbele punten. Een octet bestaat uit acht bits, wat betekent dat een octet een bereik heeft van 0 tot 255.

In essentie zorgt een IP-adres ervoor dat apparaten op het internet met elkaar kunnen communiceren door te weten waar ze gegevens naartoe moeten sturen en waar ze gegevens vandaan kunnen halen.

\subsection{Subnetten}
% TODO: Bron zoeken om te staven wat ik schrijf over IP
Een subnet (kort voor subnetwerk) is een logisch gescheiden deel van een netwerk, het verdeelt een netwerk in kleinere delen voor een efficiënter beheer. Elk subnetwerk bevat een netwerkadres, host-adressen, een broadcastadres en een netwerkmasker. Om dit te verduidelijken wordt hier het netwerk 192.168.0.0/24 ontleed om elk element uit te leggen.

\begin{itemize}
    \item Netwerkadres: Dit is het eerste adres van het netwerk (in het voorbeeld: 192.168.0.0). Dit adres dient om het subnet te identificeren.
    \item Netwerkmasker: Dankzij het netwerkmasker kan worden vastgesteld welk deel (het netwerkgedeelte) van een IP-adres het netwerk vertegenwoordigt en welk deel (het hostgedeelte) bestemd is voor individuele hosts. In het voorbeeld staat /24, wat aangeeft dat de eerste drie octetten van het IP-adres het netwerkadres vormen, terwijl het laatste octet beschikbaar is voor apparaten. Dit komt overeen met de notatie 255.255.255.0, waarbij de eerste drie octetten '255' zijn (wat betekent dat ze allemaal voor het netwerk bestemd zijn) en het laatste octet '0' is (wat beschikbaar is voor apparaten). De notatie "/24" (gekend als prefix) komt overeen met drie octetten omdat elk octet in een IP-adres bestaat uit acht bits, en "/24" betekent dat er 24 bits worden gebruikt voor het netwerkadres, waardoor er acht bits overblijven voor de hostadressen.      
    \item Broadcastadres: Dit is steeds het laatste adres van het netwerk (in het voorbeeld: 192.168.0.255). Dankzij dit adres kunnen apparaten berichten sturen naar alle apparaten die zich in dit subnetwerk bevinden.
    \item Hostadressen: Een hostadres is het uniek identificatienummer van een apparaat binnen een subnet. Met behulp van het netwerkmasker kan worden vastgesteld hoeveel hostadressen beschikbaar zijn in elk subnet. In het voorbeeld zijn er 254 beschikbare hostadressen. Het adres 0 wordt gereserveerd voor het netwerkadres en het adres 255 voor het broadcastadres. Dit betekent dat alle adressen tussen 1 en 254 beschikbaar zijn voor apparaten om te gebruiken als hun individuele hostadressen.
\end{itemize}


IP-netwerken worden door netwerkbeheerders op een logische manier opgesplitst in subnetwerken. Hierbij worden de beschikbare IP-adressen verdeeld in subnetwerken (subnet). Het eerder beschreven voorbeeldnetwerk 192.168.0.0/24 heeft een hostgedeelte van 256 adressen, waarvan twee gereserveerd zijn voor het netwerk- en broadcastadres. Dit netwerk zou men bijvoorbeeld kunnen opdelen in twee subnetwerken, netwerk A: 192.168.0.0/25 en netwerk B: 192.168.0.128/25. Deze twee netwerken hebben elk 128 adressen, waarvan ook hier telkens twee adressen gereserveerd zijn voor het netwerk- en broadcastadres.
Toestellen binnen subnet A zullen elkaar kunnen bereiken terwijl een toestel in een subnet B zonder de nodige routering geen verbinding zal kunnen maken met de toestellen in subnet A.

\section{IT Netwerken: geavanceerde concepten}
Dit volgende segment richt zich op meer geavanceerde netwerkbegrippen, zoals DNS, DHCP, IPAM, DDI, en HTTP/HTTPS-protocollen. Deze concepten vormen de ruggengraat van moderne netwerkinfrastructuren en spelen een cruciale rol in het beheer en de communicatie binnen netwerken.

\subsection{DNS}
\textcite{Mockapetris1987} schrijft dat \acrshort{dns} een systeem is dat \textit{resource records} gebruikt om onder andere vertalingen te voorzien tussen domeinnamen en IP-adressen. Een resource record is een gegevenseenheid in de DNS-database die informatie bevat over een specifiek aspect van een domeinnaam. Als voorbeeld kan je via de browser naar google surfen via het IP-adres \textit{142.251.36.35} of via domeinnaam \textit{www.google.be}. 

Zoals beschreven door \textcite{Mockapetris1987} voorziet DNS meerdere types resource records die netwerkbeheerders kunnen meegeven: 
\begin{itemize}
    \item \textbf{A}: Dit resource record beschrijft een host adres. 
    Vb. \textit{”server1.voorbeeld.com. IN A 192.168.1.1”} maakt de vertaling zodat het toestel met de domeinnaam \textit{server1.voorbeeld.com} bereikbaar is zowel via het IP-adres \textit{192.168.1.1} als via de domeinnaam. 
    \item \textbf{CNAME}: Dit resource record beschrijft de canonieke naam van een host, het wordt gebruikt om een alias of subdomein naar het hoofddomein door te verwijzen. Vb. \textit{"www.voorbeeld.com. IN CNAME server1.voorbeeld.com"} zorgt dat server1 ook bereikbaar is via "\textit{www.voorbeeld.com}".
    \item \textbf{MX}: Dit resource record is een \textit{mail exchange} record en wordt gebruikt om aan te geven welke mailservers verantwoordelijk zijn voor het ontvangen van mails binnen een domein. Vb. De DNS-server geeft aan dat \textit{”mailserver.\\voorbeeld.com”} de mailserver is door het resource record \textit{"voorbeeld.com. IN MX 10 mailserver.voorbeeld.com"}.
    %Vb. De  \textit{"voorbeeld.com. IN MX 10 mailserver.voorbeeld.com"} geeft de  mee welke server de mailserver is.
    \item \textbf{NS}: Dit resource record is een \textit{name server} record, het beschrijft welke DNS-servers verantwoordelijk zijn voor het beheren van DNS-informatie voor een domein. Vb. \textit{"voorbeeld.com. IN NS dns1.voorbeeld.com"} verwijst naar \textit{dns1} als DNS-server voor het domein “\textit{voorbeeld.com}”.
    \item \textbf{PTR}: Dit resource record is een \textit{Pointer} record, hiermee kan de DNS-server voor een gevraagd IP-adres de domeinnaam weergeven.
    %het wordt gebruikt om via IP een vertaling te vragen aan de DNS-server in plaats van via de naam.
    \item \textbf{SOA}: Dit resource record is een \textit{Start of Authority} record die belangrijke informatie bevat over de zone, zoals welke de primaire DNS-server, contactpersonen, etc. zijn.
\end{itemize}

\subsection{DHCP}
Dit protocol voorziet een framework voor het doorgeven van configuratie-informatie naar hosts (lees: computers) op het netwerk . Zo kan een computer bijvoorbeeld een IP-adres ontvangen waarmee die kan communiceren binnen het netwerk waarop die is aangesloten \autocite{Droms1997}.

Voor \acrshort{dhcp} zullen netwerkbeheerders subnets (of pools van IP-adressen) aanbieden aan de DHCP-server. Die zal gebruik maken van deze pools door (onder andere) IP-adressen uit te delen aan toestellen die verbinden op het netwerk en daarbij de DHCP-server laten weten dat ze nog geen IP-adres hebben.

\textcite{Droms1997} schrijft dat DHCP drie mechanismes gebruikt voor het uitdelen van IP-adressen:
\begin{itemize}
    \item \textbf{Automatisch}: Permanent toewijzen van een IP-adres.
    \item \textbf{Dynamisch}: IP-adres voor een bepaalde tijd toewijzen.
    \item \textbf{Manueel}: Een (door de netwerkbeheerder) vooraf bepaald IP-adres toewijzen, in vakjargon noemt met dit een IP-reservatie.
\end{itemize}

\subsection{IPAM}
Naast de vele uitdagingen die zowel DNS als DHCP met zich meebrengen, is het beheren van de beschikbare IP-adressen een belangrijk facet in het takenpakket van de netwerkbeheerder. Dit beheer wordt vaak uitgevoerd met behulp van IP Address Management (IPAM). 
Dit is een systeem waarmee netwerkbeheerders IP-adressen kunnen plannen, volgen en beheren binnen een netwerkinfrastructuur. 
Het biedt tools en functionaliteiten om de toewijzing van IP-adressen te automatiseren, wat handmatige fouten en conflicten kan voorkomen.

Een slecht beheerd netwerk kan leiden tot IP-conflicten waarbij meerdere apparaten hetzelfde IP-adres gebruiken, wat op zijn beurt dan weer kan leiden tot netwerkstoringen en onvoorspelbaar gedrag van apparaten. IPAM biedt hier een oplossing door een georganiseerd overzicht te bieden van alle toegewezen IP-adressen en subnetten in het netwerk. Hierdoor kunnen beheerders snel potentiële conflicten identificeren en proactief maatregelen nemen om ze op te lossen, wat de algehele netwerkprestaties verbetert en de operationele efficiëntie verhoogt.

\subsection{DNS, DHCP en IPAM}
De integratie van DNS, DHCP en \acrshort{ipam} in een enkel systeem wordt vaak aangeduid als \acrshort{ddi}. Dit biedt een benadering voor het beheren van de essentiële netwerkelementen die nodig zijn voor een goed functionerende infrastructuur. Door DNS, DHCP en IPAM te combineren, kunnen organisaties profiteren van een geïntegreerde aanpak voor het beheer van hun netwerkmiddelen, waardoor efficiëntie en consistentie worden bevorderd. Zoals beschreven door \textcite{Fontein2023}, leggen DDI-applicaties doorgaans de focus voornamelijk op IPAM. Het zijn toepassingen die de beschikbare IP-adressen en subnetten op een gestructureerde, overzichtelijke manier weergeven. Doorgaans is DDI geïntegreerd met de DNS- en DHCP-servers waardoor men deze componenten vanuit een centrale plek kunnen beheren. Er zijn meerdere DDI-softwarepakketten ter beschikking die aangeboden worden door bedrijven zoals Solarwinds, Infoblox, en EfficientIP. \acrshort{ugent} heeft ervoor gekozen om \acrlong{eip} aan te schaffen, dus deze bachelorproef zal hier gebruik van maken voor het automatisch beheren van het netwerk.

\subsection{HTTP/HTTPS}
Om communicatie met een \acrshort{api} mogelijk te maken wordt gebruik gemaakt van \acrfull{http}. Dat is een client-serverprotocol die communicatie mogelijk maakt op het Internet. Zoals beschreven door \textcite{Fielding2014} maakt HTTP gebruik van \acrfull{uri} om unieke web-resources te identificeren en biedt het verschillende methoden \textit{(GET, POST, PUT, DELETE)} waarmee clients acties kunnen uitvoeren op serverresources. HTTP is \textit{stateless}, elke aanvraag is onafhankelijk, en statuscodes zoals "200 OK"\ en \textit{headers} worden gebruikt om de resultaten en aanvullende informatie van serververzoeken aan te geven, waardoor een gestandaardiseerde communicatie tussen clients en servers mogelijk is.
Om deze informatieoverdracht te beveiligen, wordt \acrfull{https} gebruikt. HTTPS bouwt voort op HTTP, maar voegt een extra beveiligingslaag toe door middel van \acrshort{ssl}/\acrshort{tls}-encryptie, waardoor de uitwisseling van gegevens tussen client en server wordt versleuteld. Hierdoor worden alle IP-registraties en andere gegevens beter beschermd tegen potentiële aanvallen.

\section{Gelijkaardig Onderzoek}
In dit deel wordt aandacht besteed aan vergelijkbare projecten waarbij organisaties een overgang maken van handmatig netwerkbeheer naar geautomatiseerd beheer met behulp van een DDI-oplossing, geïmplementeerd via scripts. Ondanks uitgebreid onderzoek via Google Scholar en andere bronnen, bleek het moeilijk om veel relevante resultaten te vinden van bedrijven of instituten die hun proces gedetailleerd beschrijven van de overstap naar geautomatiseerd netwerkbeheer met behulp van een DDI-systeem. Dit benadrukt het belang van het delen van ervaringen en best practices in deze technologische overgangsfasen.

\subsection{VLAN beheer in infoblox}
Een relevant voorbeeld van een dergelijk project is beschreven door \textcite{Karaoui2023}. Zij stonden voor een vergelijkbare uitdaging en besloten hun bestaande VLAN-structuur te migreren van Excel-bestanden naar het DDI-platform van Infoblox. Een cruciaal aspect van hun aanpak was het gebruik van Python-scripts om de API van de DDI aan te spreken en zo de benodigde gegevens op de juiste locaties te plaatsen.

Om dit succesvol te implementeren, heeft \textcite{Karaoui2023} eerst een testomgeving opgezet waarin uitvoerig kon worden geëxperimenteerd en fouten konden worden opgespoord. Een belangrijk onderdeel van het proces was het converteren van de Excel-bestanden naar CSV-bestanden, waardoor de gegevens gemakkelijker te verwerken waren. Python bleek hierbij een krachtig hulpmiddel te zijn vanwege zijn vermogen om gestructureerde CSV-bestanden eenvoudig in te lezen en te bewerken.

Het verschil tussen de aanpak van UGent en die van \textcite{Karaoui2023} zal voornamelijk liggen aan de specifieke omgevingsfactoren. Bijvoorbeeld, UGent maakt gebruik van back-ups en voert tests aanvankelijk uit in de productieomgeving van EIP voordat ze naar de daadwerkelijke implementatie gaan. Aangezien het DDI-systeem van EfficientIP nog niet volledig operationeel is, worden uitgebreide tests uitgevoerd met het importeren en exporteren van alle relevante data, meer info is hierover terug te vinden in \ref{premigratie}.

