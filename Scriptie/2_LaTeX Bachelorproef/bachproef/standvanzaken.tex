\chapter{\IfLanguageName{dutch}{Stand van zaken}{State of the art}}%
\label{ch:stand-van-zaken}

% Tip: Begin elk hoofdstuk met een paragraaf inleiding die beschrijft hoe
% dit hoofdstuk past binnen het geheel van de bachelorproef. Geef in het
% bijzonder aan wat de link is met het vorige en volgende hoofdstuk.

% Pas na deze inleidende paragraaf komt de eerste sectiehoofding.

%Dit hoofdstuk bevat je literatuurstudie. De inhoud gaat verder op de inleiding, maar zal het onderwerp van de bachelorproef *diepgaand* uitspitten. De bedoeling is dat de lezer na lezing van dit hoofdstuk helemaal op de hoogte is van de huidige stand van zaken (state-of-the-art) in het onderzoeksdomein. Iemand die niet vertrouwd is met het onderwerp, weet nu voldoende om de rest van het verhaal te kunnen volgen, zonder dat die er nog andere informatie moet over opzoeken \autocite{Pollefliet2011}.

%Je verwijst bij elke bewering die je doet, vakterm die je introduceert, enz.\ naar je bronnen. In \LaTeX{} kan dat met het commando \texttt{$\backslash${textcite\{\}}} of \texttt{$\backslash${autocite\{\}}}. Als argument van het commando geef je de ``sleutel'' van een ``record'' in een bibliografische databank in het Bib\LaTeX{}-formaat (een tekstbestand). Als je expliciet naar de auteur verwijst in de zin (narratieve referentie), gebruik je \texttt{$\backslash${}textcite\{\}}. Soms is de auteursnaam niet expliciet een onderdeel van de zin, dan gebruik je \texttt{$\backslash${}autocite\{\}} (referentie tussen haakjes). Dit gebruik je bv.~bij een citaat, of om in het bijschrift van een overgenomen afbeelding, broncode, tabel, enz. te verwijzen naar de bron. In de volgende paragraaf een voorbeeld van elk.

%\textcite{Knuth1998} schreef een van de standaardwerken over sorteer- en zoekalgoritmen. Experten zijn het erover eens dat cloud computing een interessante opportuniteit vormen, zowel voor gebruikers als voor dienstverleners op vlak van informatietechnologie~\autocite{Creeger2009}.

%Let er ook op: het \texttt{cite}-commando voor de punt, dus binnen de zin. Je verwijst meteen naar een bron in de eerste zin die erop gebaseerd is, dus niet pas op het einde van een paragraaf.

%\lipsum[7-20]
Dit hoofdstuk zal eerst enkele relevante netwerkbegrippen uitleggen en eindigen met gelijkenissen met een gelijkaardig project als deze bachelorproef en wat daaruit kan worden meegenomen.
% TODO: Inleiding stand van zaken


\section{IT Netwerken: fundamenten}
Dit stuk zal enkele fundamentele begrippen uitleggen die binnen elk IT Netwerk belangrijk zijn.

\subsection{IP: Internet Protocol}
% TODO: Bron zoeken om te staven wat ik schrijf over IP
\acrfull{ip} is een set van regels die bepalen hoe een computer data zal versturen naar een andere computer. Het is het fundament van elk gestructureerd, goed functionerend en veilig netwerk. Het maakt efficiënte gegevensoverdracht mogelijk, verdeelt netwerken in beheersbare eenheden, beperkt toegang tot gevoelige informatie of systemen, identificeert services en helpt bij het oplossen van netwerkproblemen \autocite{Postel1981}. Een \acrshort{ip}-adres is volgens \textcite{Postel1981} een belangrijk mechanisme van het \acrshort{ip}. Het is een uniek identificatienummer dat aan elk apparaat is toegewezen, dat verbonden is met een computernetwerk dat het \acrshort{ip} gebruikt voor communicatie. Dit adres helpt andere computers om te weten waar ze informatie naartoe moeten sturen wanneer ze communiceren via een netwerk.
Er zijn twee versies van \acrshort{ip}-adressen, \acrshort{ip}v4 en \acrshort{ip}v6, waarvan de eerste versie bestaat uit vier groepen cijfers gescheiden door punten en de tweede versie uit acht groepen hexadecimale cijfers, gescheiden door dubbele punten. Een octet bestaat uit acht bits, wat betekent dat een octet een bereik heeft van 0 tot 255.

In essentie zorgt een \acrshort{ip}-adres ervoor dat apparaten op het internet met elkaar kunnen communiceren door te weten waar ze gegevens naartoe moeten sturen en waar ze gegevens vandaan kunnen halen.

%Dit hoofdstuk legt uit wat \acrfull{dns} en \acrfull{dhcp} is, waarom \acrshort{ipam} helpt bij het beheren van \acrshort{ip} netwerken en waarom \acrshort{http} nodig is om te communiceren met EIP. 

\subsection{Subnetten}
% TODO: Bron zoeken om te staven wat ik schrijf over IP
Een subnet (kort voor subnetwerk) is een logisch gescheiden deel van een netwerk, het verdeelt een netwerk in kleinere delen voor een efficiënter beheer. Elk subnetwerk bevat een netwerkadres, host-adressen, een broadcastadres en een netwerkmasker. Om dit te verduidelijken wordt hier het netwerk 192.168.0.0/24 ontleed om elk element uit te leggen.

\begin{itemize}
    \item Netwerkadres: Dit is het eerste adres van het netwerk (in het voorbeeld: 192.168.0.0). Dit adres dient om het subnet te identificeren.
    \item Netwerkmasker: Dankzij het netwerkmasker kan worden vastgesteld welk deel (het netwerkgedeelte) van een \acrshort{ip}-adres het netwerk vertegenwoordigt en welk deel (het hostgedeelte) bestemd is voor individuele hosts. In het voorbeeld staat /24, wat aangeeft dat de eerste drie octetten van het \acrshort{ip}-adres het netwerkadres vormen, terwijl het laatste octet beschikbaar is voor apparaten. Dit komt overeen met de notatie 255.255.255.0, waarbij de eerste drie octetten '255' zijn (wat betekent dat ze allemaal voor het netwerk bestemd zijn) en het laatste octet '0' is (wat beschikbaar is voor apparaten). De notatie "/24" komt overeen met drie octetten omdat elk octet in een \acrshort{ip}-adres bestaat uit 8 bits, en "/24" betekent dat er 24 bits worden gebruikt voor het netwerkadres, waardoor er 8 bits overblijven voor de hostadressen.      
    \item Broadcastadres: Dit is steeds het laatste adres van het netwerk (in het voorbeeld: 192.168.0.255). Dankzij dit adres kunnen apparaten berichten sturen naar alle apparaten die zich in dit subnetwerk bevinden.
    \item Hostadressen: Een hostadres is het uniek identificatienummer van een apparaat binnen een subnet. Met behulp van het netwerkmasker kan worden vastgesteld hoeveel hostadressen beschikbaar zijn in elk subnet. In het voorbeeld zijn er 254 beschikbare hostadressen. Het adres 0 wordt gereserveerd voor het netwerkadres en het adres 255 voor het broadcastadres. Dit betekent dat alle adressen tussen 1 en 254 beschikbaar zijn voor apparaten om te gebruiken als hun individuele hostadressen.
\end{itemize}


\acrshort{ip}-netwerken worden door netwerkbeheerders op een logische manier opgesplitst in subnetwerken. Hierbij worden de beschikbare \acrshort{ip}-adressen verdeeld in subnetwerken (subnet). Het eerder beschreven voorbeeldnetwerk 192.168.0.0/24 heeft een hostgedeelte van 256 adressen, waarvan 2 gereserveerd zijn voor het netwerk- en broadcastadres. Dit netwerk zou men bijvoorbeeld kunnen opdelen in twee subnetwerken, netwerk A: 192.168.0.0/25 en netwerk B: 192.168.0.128/25. Deze twee netwerken hebben elk 128 adressen, waarvan ook hier telkens 2 adressen gereserveerd zijn voor het netwerk- en broadcastadres.
Toestellen binnen subnet A zullen elkaar kunnen bereiken terwijl een toestel in een subnet B zonder de nodige routering geen verbinding zal kunnen maken met de toestellen in subnet A.

\section{IT Netwerken: geavanceerde concepten}
Dit stuk zal meer geavanceerde concepten beschrijven die nodig zijn om een netwerk beter te begrijpen en beheren.

\subsection{DNS}
\textcite{Mockapetris1987} schrijft dat \acrshort{dns} een systeem is dat \textit{resource records} gebruikt om onder andere vertalingen te voorzien tussen domeinnamen en \acrshort{ip}-adressen. Een resource record is een gegevenseenheid in de \acrshort{dns}-database die informatie bevat over een specifiek aspect van een domeinnaam. Als voorbeeld kan je via de browser naar google surfen via het \acrshort{ip}-adres \textit{142.251.36.35} of via domeinnaam \textit{www.google.be}. 

Zoals beschreven door \textcite{Mockapetris1987} voorziet \acrshort{dns} meerdere types resource records die netwerkbeheerders kunnen meegeven: 
\begin{itemize}
    \item \textbf{A}: Dit resource record beschrijft een host adres. 
    Vb. \textit{”server1.voorbeeld.com. IN A 192.168.1.1”} maakt de vertaling zodat het toestel met de domeinnaam \textit{server1.voorbeeld.com} bereikbaar is zowel via het \acrshort{ip}-adres \textit{192.168.1.1} als via de domeinnaam. 
    \item \textbf{CNAME}: Dit resource record beschrijft de canonieke naam van een host, het wordt gebruikt om een alias of subdomein naar het hoofddomein door te verwijzen. Vb. \textit{"www.voorbeeld.com. IN CNAME server1.voorbeeld.com"} zorgt dat server1 ook bereikbaar is via "\textit{www.voorbeeld.com}".
    \item \textbf{MX}: Dit resource record is een \textit{mail exchange} record en wordt gebruikt om aan te geven welke mailservers verantwoordelijk zijn voor het ontvangen van mails binnen een domein. Vb. De \acrshort{dns}-server geeft aan dat \textit{”mailserver.\\voorbeeld.com”} de mailserver is door het resource record \textit{"voorbeeld.com. IN MX 10 mailserver.voorbeeld.com"}.
    %Vb. De  \textit{"voorbeeld.com. IN MX 10 mailserver.voorbeeld.com"} geeft de  mee welke server de mailserver is.
    \item \textbf{NS}: Dit resource record is een \textit{name server} record, het beschrijft welke \acrshort{dns}-servers verantwoordelijk zijn voor het beheren van \acrshort{dns}-informatie voor een domein. Vb. \textit{"voorbeeld.com. IN NS dns1.voorbeeld.com"} verwijst naar \textit{dns1} als \acrshort{dns}-server voor het domein “\textit{voorbeeld.com}”.
    \item \textbf{PTR}: Dit resource record is een \textit{Pointer} record, hiermee kan de \acrshort{dns}-server voor een gevraagd \acrshort{ip}-adres de domeinnaam weergeven.
    %het wordt gebruikt om via \acrshort{ip} een vertaling te vragen aan de \acrshort{dns}-server in plaats van via de naam.
    \item \textbf{SOA}: Dit resource record is een \textit{Start of Authority} record die belangrijke informatie bevat over de zone, zoals welke de primaire \acrshort{dns}-server, contactpersonen, etc. zijn.
\end{itemize}

\subsection{DHCP}
Dit protocol voorziet een framework voor het doorgeven van configuratie-informatie naar hosts (lees: computers) op het netwerk . Zo kan een computer bijvoorbeeld een \acrshort{ip}-adres ontvangen waarmee die kan communiceren binnen het netwerk waarop die is aangesloten \autocite{Droms1997}.

Voor \acrshort{dhcp} zullen netwerkbeheerders subnets (of pools van \acrshort{ip}-adressen) aanbieden aan de \acrshort{dhcp}-server. Die zal gebruik maken van deze pools door (onder andere) \acrshort{ip}-adressen uit te delen aan toestellen die verbinden op het netwerk en daarbij de \acrshort{dhcp}-server laten weten dat ze nog geen \acrshort{ip}-adres hebben.

\textcite{Droms1997} schrijft dat \acrshort{dhcp} drie mechanismes gebruikt voor het uitdelen van \acrshort{ip}-adressen:
\begin{itemize}
    \item \textbf{Automatisch}: Permanent toewijzen van een \acrshort{ip}-adres.
    \item \textbf{Dynamisch}: \acrshort{ip}-adres voor een bepaalde tijd toewijzen.
    \item \textbf{Manueel}: Een (door de netwerkbeheerder) vooraf bepaald \acrshort{ip}-adres toewijzen, in vakjargon noemt met dit een \acrshort{ip}-reservatie.
\end{itemize}

\subsection{IPAM}
Naast de vele uitdagingen die zowel \acrshort{dns} als \acrshort{dhcp} met zich meebrengen, is het beheren van de beschikbare \acrshort{ip}-adressen een belangrijk facet in het takenpakket van de netwerkbeheerder. Een slecht beheerd netwerk kan leiden tot \acrshort{ip}-conflicten waarbij meerdere apparaten hetzelfde \acrshort{ip}-adres gebruiken, wat op zijn beurt dan weer kan leiden tot netwerkstoringen en onvoorspelbaar gedrag van apparaten. Daarnaast zou men ook weinig tot geen overzicht hebben van de beschikbare \acrshort{ip}-adressen en subnetten in het netwerk, waardoor het moeilijker is om de beschikbare adressen te beheren en potentiële conflicten te identificeren.

\subsection{DNS, DHCP en IPAM}
De integratie van \acrshort{dns}, \acrshort{dhcp} en \acrshort{ipam} in een enkel systeem wordt vaak aangeduid als \acrshort{ddi}. Dit biedt een benadering voor het beheren van de essentiële netwerkelementen die nodig zijn voor een goed functionerende infrastructuur. Door \acrshort{dns}, \acrshort{dhcp} en \acrshort{ipam} te combineren, kunnen organisaties profiteren van een geïntegreerde aanpak voor het beheer van hun netwerkmiddelen, waardoor efficiëntie en consistentie worden bevorderd. Zoals beschreven door \textcite{Fontein2023}, leggen \acrshort{ddi}-applicaties doorgaans de focus voornamelijk op \acrshort{ipam}. Het zijn toepassingen die de beschikbare \acrshort{ip}-adressen en subnetten op een gestructureerde, overzichtelijke manier weergeven. Doorgaans is \acrshort{ddi} geïntegreerd met de \acrshort{dns}- en \acrshort{dhcp}-servers waardoor men deze componenten vanuit een centrale plek kunnen beheren. Er zijn meerdere \acrshort{ddi}-softwarepakketten ter beschikking die aangeboden worden door bedrijven zoals Solarwinds, Infoblox, en EfficientIP. \acrshort{ugent} heeft ervoor gekozen om \acrlong{eip} aan te schaffen, dus deze bachelorproef zal hier gebruik van maken voor het automatisch beheren van het netwerk.

\subsection{HTTP/HTTPS}
Om communicatie met een \acrshort{api} mogelijk te maken wordt gebruik gemaakt van \acrfull{http}. Dit is een client-serverprotocol die communicatie mogelijk maakt op het Internet. Zoals beschreven door \textcite{Fielding2014} maakt \acrshort{http} gebruik van \acrfull{uri} om unieke web-resources te identificeren en biedt het verschillende methoden \textit{(GET, POST, PUT, DELETE)} waarmee clients acties kunnen uitvoeren op serverresources. \acrshort{http} is \textit{stateless}, elke aanvraag is onafhankelijk, en statuscodes zoals "200 OK"\ en \textit{headers} worden gebruikt om de resultaten en aanvullende informatie van serververzoeken aan te geven, waardoor een gestandaardiseerde communicatie tussen clients en servers mogelijk is.
Om deze informatieoverdracht te beveiligen, wordt \acrfull{https} gebruikt. \acrshort{https} bouwt voort op \acrshort{http}, maar voegt een extra beveiligingslaag toe door middel van \acrshort{ssl}/\acrshort{tls}-encryptie, waardoor de uitwisseling van gegevens tussen client en server wordt versleuteld. Hierdoor worden alle \acrshort{ip}-registraties en andere gegevens beter beschermd tegen potentiële aanvallen.

\section{Gelijkaardig Onderzoek}
In dit onderdeel wordt gekeken naar gelijkaardige projecten waarbij organisaties via scripts hun manueel netwerkbeheer automatiseren via het implementeren van een \acrshort{ddi}-oplossing.

\subsection{VLAN beheer in infoblox}
Zoals beschreven door \textcite{Karaoui2023} hadden ze hun bestaande VLAN-structuur overgezet van excel-bestanden naar \acrshort{ddi} Infoblox. Ze gebruikten python scripts om de \acrshort{api} van de \acrshort{ddi} aan te roepen en de data op de juiste locatie te plaatsen. Hiervoor hadden ze een testomgeving opgezet en hadden ze hun excel-bestanden geconverteerd naar CSV-bestanden. Het voordeel hiervan is dat python een CSV-bestand gemakkelijk kan uitlezen en bewerken aangezien deze steeds een vaste structuur hebben.

Het verschil tussen hun aanpak zal voornamelijk aan de omgeving liggen, \acrshort{ugent} maakt gebruik van back-ups waarbij testen initieel in de productieomgeving van \acrshort{eip} gedaan worden. Aangezien \acrshort{eip} nog niet in productie was werden uitvoerig testen gedaan met het importen en exporteren van alle data, meer info is hierover terug te vinden in \ref{premigratie}.

