%%=============================================================================
%% Samenvatting
%%=============================================================================

% TODO: De "abstract" of samenvatting is een kernachtige (~ 1 blz. voor een
% thesis) synthese van het document.
%
% Een goede abstract biedt een kernachtig antwoord op volgende vragen:
%
% 1. Waarover gaat de bachelorproef?
% 2. Waarom heb je er over geschreven?
% 3. Hoe heb je het onderzoek uitgevoerd?
% 4. Wat waren de resultaten? Wat blijkt uit je onderzoek?
% 5. Wat betekenen je resultaten? Wat is de relevantie voor het werkveld?
%
% Daarom bestaat een abstract uit volgende componenten:
%
% - inleiding + kaderen thema
% - probleemstelling
% - (centrale) onderzoeksvraag
% - onderzoeksdoelstelling
% - methodologie
% - resultaten (beperk tot de belangrijkste, relevant voor de onderzoeksvraag)
% - conclusies, aanbevelingen, beperkingen
%
% LET OP! Een samenvatting is GEEN voorwoord!

%%---------- Nederlandse samenvatting -----------------------------------------
%
% TODO: Als je je bachelorproef in het Engels schrijft, moet je eerst een
% Nederlandse samenvatting invoegen. Haal daarvoor onderstaande code uit
% commentaar.
% Wie zijn bachelorproef in het Nederlands schrijft, kan dit negeren, de inhoud
% wordt niet in het document ingevoegd.

\IfLanguageName{english}{%
\selectlanguage{dutch}
\chapter*{Samenvatting}
\lipsum[1-4]
\selectlanguage{english}
}{}

%%---------- Samenvatting -----------------------------------------------------
% De samenvatting in de hoofdtaal van het document

\chapter*{\IfLanguageName{dutch}{Samenvatting}{Abstract}}
Het beheren van netwerken en het reserveren van netwerkadressen gebeurt momenteel grotendeels handmatig, wat inefficiënt is als proces, en tevens gevoelig voor menselijke fouten. 

Deze bachelorproef richt zich op het uitwerken van een innovatieve, geautomatiseerde aanpak voor het beheren van netwerken en het
toewijzen van netwerkadressen met behulp van scripts. 
Het doel van deze bachelorproef is tweeledig, namelijk (1) het creëren van een tussenlaag van scripts boven een bestaand beheerprogramma voor netwerken, en (2) het nagaan van de impact hiervan op het tijdverbruik voor het toevoegen, wijzigen en verwijderen van nieuwe IP-reserveringen in vergelijking met het huidige handmatige beheerproces. Als resultaat van de bachelorproef kan men via een webpagina die beschikbaar is binnen het bedrijfsnetwerk, mits toestemming van de netwerkbeheerder, eenvoudig netwerkadresreservaties aanmaken, wijzigen of verwijderen.
De webpagina zal de scripts, die binnen het project geschreven worden, aanroepen om de nodige gegevens op de juiste manier aan te leveren aan het beheerprogramma. 

Het verminderen van handmatig beheer van netwerkconfiguraties als resultaat van deze bachelorproef, zal leiden tot efficiëntiewinsten, tijdsbesparingen, een vereenvoudigde aanpak en minder fouten. 
