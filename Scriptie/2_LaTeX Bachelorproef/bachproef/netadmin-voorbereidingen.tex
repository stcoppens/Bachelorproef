%%=============================================================================
%% netadmin-voorbereidingen
%%=============================================================================

\chapter{Netadmin - Voorbereidingen}%
\label{ch:netadmin-voorbereidingen}

Voorafgaand aan de ontwikkeling van de nieuwe website worden verschillende voorbereidingen getroffen om een solide basis te leggen voor het project. Eerst worden er metingen genomen om na te gaan hoeveel tijd de netwerkbeheerder dagelijks moet investeren om de registraties van die dag te behandelen. Daarna wordt nagegaan wat van de nieuwe website verwacht wordt, hoe die geïmplementeerd kan worden in de huidige infrastructuur, en hoe de ontwikkeling zal aangepakt worden.

\section{Tijdmetingen IP-registraties}
Zoals beschreven in hoofdstuk \ref{ch:voorgeschiedenis}, ontvangen de netwerkbeheerders van UGent bijna dagelijks meerdere e-mails met IP-registraties van gebruikers. Nadat de netwerkbeheerder heeft beoordeeld of de registratie geschikt is voor uitvoering, moet deze inloggen op de server waar de subnetbestanden zich bevinden. Vervolgens verkrijgt de netwerkbeheerder de juiste privileges door over te schakelen naar de gebruiker die wordt gebruikt voor registraties, om uiteindelijk naar de map te navigeren waar de subnetbestanden zich bevinden en de benodigde acties uit te voeren om de registratie te voltooien. Als het subnet niet in de e-mail staat, zoekt de netwerkbeheerder eerst het subnet op, voert het commando in, plakt de inhoud van de e-mail in het bestand en slaat het op als er geen verdere wijzigingen nodig zijn. Indien het subnet vol is, voert de netwerkbeheerder opruimacties uit.

Na elke registratie wordt de e-mail ook verplaatst naar een aparte map in de mailbox. Deze map wordt later gebruikt om te achterhalen wie welke registratie heeft behandeld en wie die heeft aangemaakt. Dit wordt gedaan om een duidelijk traceerbaarheidssysteem te handhaven voor alle registratieactiviteiten.

Om het dagelijkse tijdsverbruik te meten, is besloten om gedurende 13 opeenvolgende werkdagen alle registraties dagelijks te bundelen en deze te laten uitvoeren door de waarnemer, de heer S. Coppens. Bij aanvang van de metingen is de waarnemer al ingelogd op de juiste server in de juiste map en onder de juiste gebruiker. Elke registratie wordt afzonderlijk gemeten en genoteerd in een Excel-bestand, dat zal worden gebruikt om in de conclusie van de bachelorproef te bepalen hoeveel tijd er kan worden bespaard door de registraties te automatiseren. Deze metingen zijn terug te vinden in Bijlage\_1\_metingen\_preimplementatie.xlsx.

\section{Analyse van de oude registratie website}
Voordat het ontwikkelingsproces van start gaat, wordt de bestaande website grondig geëvalueerd om te bepalen welke functionaliteiten kunnen worden overgenomen in de nieuwe website. Dit omvat een diepgaande analyse van de bestaande processen en het identificeren van de vereisten voor de nieuwe website. Flowcharts worden gemaakt om de onderliggende processen te visualiseren en een duidelijk beeld te krijgen van de vereiste functionaliteiten. Deze flowcharts zijn niet definitief en dienen slechts als leidraad om enkele belangrijke stappen weer te geven.

\subsection{Flowcharts van gewenste functionaliteiten}
Er worden vijf flowcharts gemaakt voor acties die mogelijk moeten zijn op de nieuwe website.
De eerste functie die elke ingelogde gebruiker op de website zal moeten kunnen is het opvragen van bestaande registraties, de flowchart hiervoor is terug te vinden in figuur  \ref{fig:flowchart_hostinfo_opvragen}. Eens de gebruiker de registratie gevonden heeft die aangepast moet worden, kan de registratie aangepast worden zoals weergegeven in figuur \ref{fig:flowchart_registratie_wijzigen}. Naast het wijzigen en verwijderen van registraties zal een gebruiker ook moeten kunnen aanmaken, een mogelijke flow is terug te vinden in figuur \ref{fig:flowchart_nieuwe_registratie}.
De netwerkbeheerders zullen een eigen pagina nodig hebben waar ze de nieuwe/gewijzigde registraties kunnen controleren en al dan niet goed- of afkeuren. Een mogelijke methode hiervoor is weergegeven in \ref{fig:flowchart_controle_Wijzigingen}. In sommige gevallen kan het voorkomen dat een subnet vol zit, in dat geval zal de netwerkbeheerder oudere registraties moeten kunnen verwijderen. Deze flowchart is terug te vinden in \ref{fig:flowchart_subnet_opkuisen}.

\begin{figure}[H]
	\includegraphics[width=16cm]{Flowchart_Hostinfo opvragen.jpg}
	\caption{Hostinfo opvragen}
	\label{fig:flowchart_hostinfo_opvragen}
\end{figure}
\begin{figure}[H]
    \includegraphics[width=16cm]{Flowchart_Registratie verwijderen of wijzigen.jpg}
    \caption{Registratie wijzigen}
    \label{fig:flowchart_registratie_wijzigen}
\end{figure}
\begin{figure}[H]
    \includegraphics[width=16cm]{Flowchart_Nieuwe registratie.jpg}
    \caption{Nieuwe registratie aanmaken}
    \label{fig:flowchart_nieuwe_registratie}
\end{figure}
\begin{figure}[H]
    \includegraphics[width=16cm]{Flowchart_Controle IPAM wijzigingen.jpg}
    \caption{Controle wijzigingen}
    \label{fig:flowchart_controle_Wijzigingen}
\end{figure}
\begin{figure}[H]
    \includegraphics[width=16cm]{Flowchart_Subnet opkuis.jpg}
    \caption{Subnet opkuisen}
    \label{fig:flowchart_subnet_opkuisen}
\end{figure}

\section{Integratie binnen de Infrastructuur van UGent}
De integratie van de nieuwe website binnen de infrastructuur van UGent vereist specifieke overwegingen. Eerdere pogingen om de netadmin website te vernieuwen zijn mislukt vanwege de complexiteit van de materie en de verouderde code. Het is essentieel dat de nieuwe website enkele jaren kan meegaan en kan voldoen aan de eisen van UGent. Er wordt besloten om PHP te gebruiken voor de ontwikkeling, gezien de mogelijkheid om Python-scripts aan te roepen, waardoor het project kan worden geïntegreerd in de bestaande infrastructuur.

\section{Agile Ontwikkelinsaanpak}
Er zijn reeds wekelijkse vergaderingen met enkele stakeholders die het netwerkteam zijn om hun actief te betrekken bij de ontwikkeling en hun de huidige status mee te geven. Daarnaast worden er ook wekelijkse vergaderingen opgezet met collega A. De Keyzer, die ervaring heeft met PHP en eerder aan vergelijkbare projecten heeft gewerkt. In deze vergaderingen wordt de voortgang besproken en worden er  beslissingen genomen over de ontwikkeling van de website.

Om flexibel te kunnen reageren op veranderende eisen en prioriteiten, wordt besloten om een agile ontwikkelingsaanpak te hanteren. Dit betekent dat het ontwikkelproces iteratief en incrementeel verloopt, waarbij regelmatig updates worden uitgebracht op basis van user stories. Daarnaast wordt een GitHub-repository opgezet om de broncode te beheren en een development MySQL-database om de ontwikkeling te ondersteunen en informatie persistent te bewaren.

Al deze voorbereidingen vormen de basis voor een gestructureerde en georganiseerde aanpak van de ontwikkeling van de nieuwe website.
