%%=============================================================================
%% netadmin-voorbereidingen
%%=============================================================================

\chapter{Netadmin - Voorbereidingen}%
\label{ch:netadmin-voorbereidingen}

Voorafgaand aan de ontwikkeling van de nieuwe website worden verschillende voorbereidingen getroffen om een solide basis te leggen voor het project. Hieronder worden de belangrijkste stappen beschreven die worden genomen:

\section{Analyse van de oude registratie website}
Voordat het ontwikkelingsproces van start gaat, wordt de bestaande website grondig geëvalueerd om te bepalen welke functionaliteiten kunnen worden overgenomen in de nieuwe website. Dit omvat een diepgaande analyse van de bestaande processen en het identificeren van de vereisten voor de nieuwe website. Flowcharts worden gemaakt om de onderliggende processen te visualiseren en een duidelijk beeld te krijgen van de vereiste functionaliteiten.

% TODO: Toevoegen flowcharts en kort bespreken

\section{Integratie binnen de Infrastructuur van UGent}
De integratie van de nieuwe website binnen de infrastructuur van UGent vereist specifieke overwegingen. Eerdere pogingen om de netadmin website te vernieuwen zijn mislukt vanwege de complexiteit van de materie en de verouderde code. Het is essentieel dat de nieuwe website enkele jaren kan meegaan en kan voldoen aan de eisen van UGent. Er wordt besloten om PHP te gebruiken voor de ontwikkeling, gezien de mogelijkheid om Python-scripts aan te roepen, waardoor het project kan worden geïntegreerd in de bestaande infrastructuur.

Agile Ontwikkelingsaanpak:
\section{Agile Ontwikkelinsaanpak}
Er zijn reeds wekelijkse vergaderingen met enkele stakeholders die het netwerkteam zijn om hun actief te betrekken bij de ontwikkeling en hun de huidige status mee te geven. Daarnaast worden er ook wekelijkse vergaderingen opgezet met collega A. De Keyzer, die ervaring heeft met PHP en eerder aan vergelijkbare projecten heeft gewerkt. In deze vergaderingen wordt de voortgang besproken en worden er  beslissingen genomen over de ontwikkeling van de website.

Om flexibel te kunnen reageren op veranderende eisen en prioriteiten, wordt besloten om een agile ontwikkelingsaanpak te hanteren. Dit betekent dat het ontwikkelproces iteratief en incrementeel verloopt, waarbij regelmatig updates worden uitgebracht op basis van user stories. Daarnaast wordt een GitHub-repository opgezet om de broncode te beheren en een development MySQL-database om de ontwikkeling te ondersteunen en informatie persistent te bewaren.

Al deze voorbereidingen vormen de basis voor een gestructureerde en georganiseerde aanpak van de ontwikkeling van de nieuwe website.
