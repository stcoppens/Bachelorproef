%%=============================================================================
%% Inleiding
%%=============================================================================

\chapter{\IfLanguageName{dutch}{Inleiding}{Introduction}}%
\label{ch:inleiding}
In de snel evoluerende wereld van technologie is het doeltreffend beheren van netwerken cruciaal geworden voor het succes van organisaties. Niettemin blijft het handmatig beheren van netwerkconfiguraties, waaronder het toewijzen van specifieke netwerkadressen, een tijdrovend en foutgevoelig proces.

Om deze uitdagingen aan te pakken, zijn er geautomatiseerde netwerkbeheersystemen om het uitdelen van specifieke netwerkadressen te beheren. Een goed uitgewerkt beheersysteem kan een grote meerwaarde bieden voor elk netwerk doordat deze een overzicht kan geven van alle netwerkadressen die in gebruik zijn en hoeveel netwerkadressen er nog beschikbaar zijn per segment van het netwerk.

Dit onderzoek zal scripts voorzien die, via een webportaal en de goedkeuring van netwerkbeheerders, gebruikers zullen toelaten te communiceren met een dergelijk netwerkbeheersysteem om netwerkadres reservaties te maken, te wijzigen of te verwijderen. Dankzij autorisatie zullen gebruikers op het webportaal enkel wijzigingen kunnen aanbrengen voor de netwerken waarvoor de gebruiker gemachtigd is. Nadat de netwerkbeheerder wijzigingen via hetzelfde webportaal goedkeurt, zullen de scripts in werking treden om de nodige aanpassingen binnen het gebruikte beheersysteem te maken.

\section{\IfLanguageName{dutch}{Probleemstelling}{Problem Statement}}%
\label{sec:probleemstelling}
Bij het begin van de bachelorproef worden handmatige registraties van apparaten binnen het netwerk van de Universiteit Gent bijgehouden via tekstbestanden met een vooraf bepaalde structuur. Medewerkers van de Universiteit Gent kunnen via een verouderde webpagina registraties aanmaken voor hun apparaten wanneer dit nodig is. Niet alle medewerkers weten in welk deel van het netwerk de registratie thuishoort, waardoor ze soms enkele belangrijke velden niet kunnen invullen. Indien dit voorkomt, kan het voor de netwerkbeheerder moeilijk zijn om te bepalen in welk netwerksegment de registratie hoort, wat soms tot fouten leidt. Zodra de medewerker de registratie indient, wordt deze als een e-mail naar de netwerkbeheerders gestuurd, die tijd moeten besteden om ervoor te zorgen dat de registratie correct in het juiste tekstbestand wordt geplaatst.

\section{\IfLanguageName{dutch}{Onderzoeksvraag}{Research question}}%
\label{sec:onderzoeksvraag}
Dit onderzoek stelt twee onderzoeksvragen:
\begin{enumerate}
    \item Hoeveel tijd spenderen netwerkbeheerders van Universiteit Gent dagelijks aan het aanmaken, wijzigen en/of verwijderen van manuele netwerkregistraties van toestellen op het netwerk?
    \item Hoe kan men via een softwareportaal van een netwerkbeheersysteem, een eenvoudig te gebruiken platform maken waarop medewerkers van Universiteit Gent netwerkregistraties kunnen aanmaken?
    \begin{enumerate}
        \item Hoe kunnen de netwerkbeheerders van Universiteit Gent registraties via dit platform beheren?
    \end{enumerate}
\end{enumerate}

\section{\IfLanguageName{dutch}{Onderzoeksdoelstelling}{Research objective}}%
\label{sec:onderzoeksdoelstelling}
Deze bachelorproef heeft als doel een webpagina te ontwerpen die via scripts zal communiceren met een nieuw netwerkbeheersysteem. De website biedt medewerkers van de Universiteit Gent de mogelijkheid om informatie van bestaande registraties op te halen, deze te wijzigen of te verwijderen, en nieuwe registraties aan te maken. Na goedkeuring door de netwerkbeheerder van de Universiteit Gent zullen deze registraties op een correcte en veilige manier worden overgebracht naar het nieuwe netwerkbeheersysteem. Daarnaast zal de bachelorproef ook onderzoeken hoeveel tijd kan worden bespaard door het proces van het toevoegen van registraties aan het netwerkbeheersysteem te automatiseren.

\section{\IfLanguageName{dutch}{Opzet van deze bachelorproef}{Structure of this bachelor thesis}}%
\label{sec:opzet-bachelorproef}
De rest van deze bachelorproef is als volgt opgebouwd:

In Hoofdstuk~\ref{ch:stand-van-zaken} wordt een overzicht gegeven van de stand van zaken binnen het onderzoeksdomein, op basis van een literatuurstudie.

In Hoofdstuk~\ref{ch:voorgeschiedenis} wordt het duidelijk welke stappen reeds ondernomen zijn binnen Universiteit Gent in hun migratie naar een nieuw netwerkbeheersysteem.

In Hoofdstuk~\ref{ch:methodologie} wordt de methodologie toegelicht en worden de gebruikte onderzoekstechnieken besproken om een antwoord te kunnen formuleren op de onderzoeksvragen.

In Hoofdstuk~\ref{ch:netadmin-voorbereidingen} wordt beschreven hoe de opbouw en implementatie van de website werd voorbereid om binnen de infrastructuur van Universiteit Gent te passen.

In Hoofdstuk~\ref{ch:netadmin-website-ontwikkeling} worden de belangrijkste onderdelen van de website en hun functie beschreven.

Hoofdstuk~\ref{ch:conclusie}geeft uiteindelijk de conclusie en formuleert een antwoord op de onderzoeksvragen, waarbij ook een aanzet wordt gegeven voor toekomstig onderzoek binnen dit domein.