%%=============================================================================
%% Voorwoord
%%=============================================================================

\chapter*{\IfLanguageName{dutch}{Woord vooraf}{Preface}}%
\label{ch:voorwoord}

Dit werk is het resultaat van mijn diepgaande interesse en betrokkenheid bij IT-netwerken, die zich ontwikkelde vanaf mijn opleiding in systeembeheer bij VDAB tot mijn huidige rol als netwerkbeheerder binnen het netwerkteam van de Universiteit Gent.
Mijn motivatie voor dit onderwerp ontstond uit mijn passie voor IT-netwerken en mijn streven naar een beter begrip van de complexe IT-infrastructuur die onze moderne samenleving ondersteunt. 
Met een achtergrond in Cisco-certificeringen en een toewijding aan mijn huidige opleiding Bachelor in de Toegepaste Informatica aan de Hogeschool Gent, zag ik in de uitdagingen van de Universiteit Gent op het gebied van netwerkautomatisering en IP-registratie een kans om mijn kennis en vaardigheden verder te ontwikkelen en toe te passen tijdens mijn studie.
Tijdens mijn onderzoek stuitte ik op verschillende uitdagingen, variërend van het verbeteren van mijn programmeervaardigheden in Python, tot het leiden van een project voor het opzetten van een nieuwe registratietool. Het implementeren van de vereisten van het netwerkteam, met complexe permissies en functionele eisen, bracht zijn eigen uitdagingen met zich mee. Desondanks heb ik deze obstakels met vastberadenheid aangepakt en ben ik gegroeid als professional in het proces.
Ik ben dankbaar voor de steun die ik heb ontvangen tijdens dit project. Lien en Doesjka verdienen speciale vermelding voor hun onschatbare steun, advies en feedback gedurende de gehele opleiding. Ook wil ik zeker mijn collega en copromotor Anne-Mie bedanken voor haar voortdurende begeleiding, advies en haar uitgebreide kennis over IT-netwerken, die van onschatbare waarde was bij het aanpakken van de complexe uitdagingen in dit onderzoek. Verder gaat mijn dank uit naar collega Arne voor zijn waardevolle bijdrage aan het project, met name bij het mee nadenken hoe we bepaalde obstakels in de registratietool kunnen oplossen, en zeker ook naar Sion voor zijn rol als promotor, die me steeds heeft voorzien van goede feedback en begeleiding.
Ten slotte hoop ik dat dit werk niet alleen een bijdrage levert aan mijn persoonlijke groei, maar ook aan het begrip en inzicht van de lezer in de complexe wereld van IT-netwerken.
