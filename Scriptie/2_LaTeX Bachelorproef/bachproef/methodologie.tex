%%=============================================================================
%% Methodologie
%%=============================================================================

\chapter{\IfLanguageName{dutch}{Methodologie}{Methodology}}%
\label{ch:methodologie}

Net als de meeste onderzoeken begint dit onderzoek met een literatuurstudie, die te vinden is in hoofdstuk \ref{ch:stand-van-zaken}.

Na deze literatuurstudie werden enkele voorbereidingen getroffen voordat het ontwikkelen van de nieuwe website van start ging. Deze voorbereidingen omvatten het maken van de nodige flowcharts die gebaseerd zijn op de acties die gebruikers op de website ondernemen, het onderzoeken van de implementatiemogelijkheden van de nieuwe website binnen de infrastructuur van UGent, en het identificeren van collega's die kunnen helpen bij de ontwikkeling van de front-end. Meer informatie hierover is te vinden in hoofdstuk \ref{ch:netadmin-voorbereidingen}.

Na deze voorbereidingen zijn er wekelijkse statusvergaderingen opgezet met S. Coppens, die verantwoordelijk is voor het schrijven van de back-end en het overzien van het project, A. De Keyzer, die de front-end ontwikkelt, en eventuele stakeholders zoals Mevr. A. Vandermeeren, die leiding geeft aan EfficientIP. Naast deze vergaderingen is er ook regelmatig informeel contact om ideeën uit te wisselen en eventuele wijzigingen aan de voorgestelde plannen te bespreken.

Er is een eerste versie van de front-end van de website zowel lokaal als op een Apache-server in de testomgeving van UGent opgezet. Deze eerste versie is gebaseerd op de besproken ideeën uit de voorbereidingsfase, de huidige netadmin-pagina en de bestaande huisstijl van UGent. De initiële versie omvat twee pagina's: een pagina om bestaande registraties op te vragen uit EfficientIP en een pagina om een nieuwe registratie aan te maken.
\begin{itemize}
    \item Registraties opzoeken: Deze pagina biedt de ingelogde gebruiker de mogelijkheid om bestaande registraties op te vragen uit EfficientIP en deze eventueel te wijzigen of te verwijderen.
    \item Nieuwe registratie: Deze pagina stelt de ingelogde gebruiker in staat om een nieuwe registratie aan te maken.
\end{itemize}

Daarnaast is er ook een MySQL-database opgezet in de testomgeving waarin zowel nieuwe registraties als gewijzigde en verwijderde registraties worden opgeslagen. Op de Apache-server wordt het script processChanges.py om de zoveel uur automatisch gestart, zodat alle goedgekeurde, onverwerkte wijzigingen worden geüpload naar EfficientIP.

Meer informatie over de verschillende versies van de website en de bijbehorende scripts is te vinden in hoofdstuk \ref{ch:netadmin-website-ontwikkeling}.